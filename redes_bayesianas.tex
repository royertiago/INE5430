\documentclass{article}
\usepackage[utf8]{inputenc}
\usepackage[brazil]{babel}
\usepackage{amsmath}
\usepackage{amssymb}
\usepackage{tikz}

\begin{document}

\title{
    INE5430 --- Inteligência Artificial \\
    Raciocínio Probabilístico --- Redes Bayesianas
}
\author{
    Tiago Royer - 12100776
}
\date{28 de Maio de 2015}
\maketitle

\section{Introdução}

O objetivo do trabalho é estudar o avistamento de zumbis no campus da UFSC.
Nos foi dado a rede bayesiana apresentada na figura \ref{rede_bayesiana}.

\begin{figure}[h]
    \centering
    \begin{tikzpicture}
        \begin{scope}[every node/.style={circle, draw, minimum size = 1cm}]
            % Minimum size é para que todos os nós possuam o mesmo tamanho.
            \node (Hp) at (0, 0) {Hp};
            \node (Mg) at (1, 1.5) {Mg};
            \node (Zb) at (2, 0) {Zb};
            \node (Vr) at (3, 1.5) {Vr};
            \node (Tp) at (4, 0) {Tp};
        \end{scope}

        \begin{scope}[>=latex, thick]
            \draw[->] (Mg) -- (Hp);
            \draw[->] (Mg) -- (Zb);
            \draw[->] (Vr) -- (Zb);
            \draw[->] (Vr) -- (Tp);
        \end{scope}
    \end{tikzpicture}
    \caption{
        Rede bayesiana usada como base no trabalho.
    }
    \label{rede_bayesiana}
\end{figure}

\end{document}
