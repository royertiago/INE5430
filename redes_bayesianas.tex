\documentclass{article}
\usepackage[utf8]{inputenc}
\usepackage[brazil]{babel}
\usepackage{amsmath}
\usepackage{amssymb}
\usepackage{tikz}
\usetikzlibrary{positioning}
\usepackage{booktabs}

\usepackage{amsthm}
\newtheorem{question}{Questão}

\newcommand{\Mg}{\mathrm{Mg}}
\newcommand{\Vr}{\mathrm{Vr}}
\newcommand{\Hp}{\mathrm{Hp}}
\newcommand{\Zb}{\mathrm{Zb}}
\newcommand{\Tp}{\mathrm{Tp}}

\begin{document}

\title{
    INE5430 --- Inteligência Artificial \\
    Raciocínio Probabilístico --- Redes Bayesianas
}
\author{
    Tiago Royer - 12100776
}
\date{28 de Maio de 2015}
\maketitle

\section{Introdução}

O objetivo do trabalho é estudar o avistamento de zumbis no câmpus da UFSC.
Nos foi dado a rede bayesiana apresentada na figura \ref{rede_bayesiana}.

\begin{figure}[h]
    \makebox[\textwidth][c]{
        % A figura é mais larga que o texto,
        % portanto o comando padrão \centering não funciona.
        % Este makebox cria uma caixa que, do ponto de vista do resto do LaTeX,
        % terá largura \textwidth;
        % mas qualquer conteúdo dentro da caixa será alinhado ao centro,
        % por mais que extrapole a largura especificada.
        \begin{tikzpicture}
            \begin{scope}[every node/.style={circle, draw, minimum size = 1cm}]
                % Minimum size é para que todos os nós possuam o mesmo tamanho.
                \node (Hp) at (0, 0) {$\Hp$};
                \node (Mg) at (1, 1.5) {$\Mg$};
                \node (Zb) at (2, 0) {$\Zb$};
                \node (Vr) at (3, 1.5) {$\Vr$};
                \node (Tp) at (4, 0) {$\Tp$};
            \end{scope}

            \begin{scope}[>=latex, thick]
                \draw[->] (Mg) -- (Hp);
                \draw[->] (Mg) -- (Zb);
                \draw[->] (Vr) -- (Zb);
                \draw[->] (Vr) -- (Tp);
            \end{scope}

            \begin{scope}[every node/.style={node distance = 1mm}]
                \node[above left = of Mg] {
                    \begin{tabular}{c c} \toprule
                        $\Mg$ = S & $\Mg$ = N \\ \midrule
                        0.1 & 0.9 \\ \bottomrule
                    \end{tabular}
                };

                \node[above right = of Vr] {
                    \begin{tabular}{c c} \toprule
                        $\Vr$ = S & $\Vr$ = N \\ \midrule
                        0.2 & 0.8 \\ \bottomrule
                    \end{tabular}
                };

                \node[left = of Hp] {
                    \begin{tabular}{c c c} \toprule
                        & $\Hp$ = S & $\Hp$ = N \\ \cmidrule{2-3}
                        $\Mg$ = S & 0.8 & 0.2 \\
                        $\Mg$ = N & 0.7 & 0.3 \\
                        \bottomrule
                    \end{tabular}
                };

                \node[right = of Tp] {
                    \begin{tabular}{c c c} \toprule
                        & $\Tp$ = S & $\Tp$ = N \\ \cmidrule{2-3}
                        $\Vr$ = S & 0.3 & 0.7 \\
                        $\Vr$ = N & 0.1 & 0.9 \\
                        \bottomrule
                    \end{tabular}
                };

                \node[below = 5mm of Zb] {
                    \begin{tabular}{c c c c} \toprule
                        & & $\Zb$ = S & $\Zb$ = N \\ \cmidrule{3-4}
                        $\Mg$ = S & $\Vr$ = S & 0.6 & 0.4 \\
                        $\Mg$ = S & $\Vr$ = N & 0.5 & 0.5 \\
                        $\Mg$ = N & $\Vr$ = S & 0.4 & 0.6 \\
                        $\Mg$ = N & $\Vr$ = N & 0.01 & 0.99 \\
                        \bottomrule
                    \end{tabular}
                };
            \end{scope}
        \end{tikzpicture}
    } % \makebox[\textwidth][c]
    \caption{
        Rede bayesiana usada como base no trabalho.
    }
    \label{rede_bayesiana}
\end{figure}

\begin{itemize}
    \item $\Mg$ representa o evento de um feitiço mágico ser lançado;
    \item $\Vr$ representa a existência de um surto viral de H1Z1;
    \item $\Hp$ é o evento de avistar um hipogrifo no câmpus;
    \item $\Zb$ representa o avistamento de zumbis; e
    \item $\Tp$ é a aparição de viajantes no tempo.
\end{itemize}

\section{Exercícios}

\begin{question}
    Qual a probabilidade de nenhum zumbi ser avistado,
    dado que não houve surto viral nem lançamento de feitiços?
\end{question}

Basta olhar na tabela logo abaixo de $\Zb$.
A quarta linha diz que,
sob a hipótese ``$\Mg$ = N'' e ``$\Vr$ = N'',
a chance de não avistarmos zumbis é de 99\%.

\begin{equation*}
    P(\Zb = N \mid \Mg = N \wedge \Vr = N) = 0.99
\end{equation*}

Esta era a única questão ``trivial'' da lista,
todas as demais preciam ser resolvidas
usando probabilidade condicional ou o teorema de Bayes.

\begin{question}
    Qual a probabilidade de os cinco eventos ocorrerem simultaneamente?
\end{question}

\begin{align*}
    P (&\Mg = S \wedge \Hp = S \wedge \Zb = S \wedge \Vr = S \wedge \Tp = S) = \\
        = & P(\Mg = S) * P(\Hp = S \mid \Mg = S) * {}\\
          & P(\Zb = S \mid \Mg = S \wedge \Vr = S) * {}\\
          & P(\Vr = S) * P(\Tp = S \mid \Vr = S) \\
        = & 0.1 * 0.8 * 0.6 * 0.8 * 0.3\\
        = & 0.01152.
\end{align*}

\begin{question}
    Qual a probabilidade de se ver um zumbi?
\end{question}

\begin{align*}
    P (\Zb = S) =
        & P(\Zb = S \mid \Mg = S \wedge \Vr = S ) * P(\Mg = S) * P(\Vr = S) + {} \\
        & P(\Zb = S \mid \Mg = S \wedge \Vr = N ) * P(\Mg = S) * P(\Vr = N) + {} \\
        & P(\Zb = S \mid \Mg = N \wedge \Vr = S ) * P(\Mg = N) * P(\Vr = S) + {} \\
        & P(\Zb = S \mid \Mg = N \wedge \Vr = N ) * P(\Mg = N) * P(\Vr = N) \\
    = & 0.6*0.1*0.2 \ +\  0.5*0.1*0.8 \ +\  0.4*0.9*0.2 \ +\  0.01*0.9*0.8 \\
    = & 0.1312.
\end{align*}

\begin{question}
    Qual a probabilidade de se ver um zumbi quando estiver havendo um surto viral?
\end{question}

\begin{align*}
    P (\Zb = S \mid \Vr = S) =
        & P(\Zb = S \mid \Mg = S \wedge \Vr = S ) * P(\Mg = S) + {} \\
        & P(\Zb = S \mid \Mg = N \wedge \Vr = S ) * P(\Mg = N) \\
    = & 0.6*0.1 \ +\  0.4*0.9 \\
    = & 0.42.
\end{align*}

\begin{question}
    Qual a probabilidade de se ver um hipogrifo após ver um zumbi?
\end{question}

Primeiro, calcularemos a probabilidade de vermos zumbis
após um feitiço mágico ser lançado,
e também caso nenhum feitiço mágico seja lançado.

\begin{align*}
    P (\Zb = S \mid \Mg = S) =
        & P(\Zb = S \mid \Mg = S \wedge \Vr = S ) * P(\Vr = S) + {} \\
        & P(\Zb = S \mid \Mg = S \wedge \Vr = N ) * P(\Vr = N) \\
    = & 0.6*0.2 \ +\  0.5*0.8 \\
    = & 0.52; \\
    P (\Zb = S \mid \Mg = N) =
        & P(\Zb = S \mid \Mg = N \wedge \Vr = S ) * P(\Vr = S) + {} \\
        & P(\Zb = S \mid \Mg = N \wedge \Vr = N ) * P(\Vr = N) \\
    = & 0.4*0.2 \ +\  0.01*0.8 \\
    = & 0.088.
\end{align*}

Agora, usando o princípio da independência condicional,
podemos resolver o problema.

\begin{align*}
    P( \Hp = S \mid \Zb = S) =&
        \Big( P( \Hp = S \wedge \Zb = S \wedge \Mg = S) + {} \\
        & P( \Hp = S \wedge \Zb = S \wedge \Mg = N) \Big) / P(\Zb = S)\\
        =& \Big( P(\Mg = S) * P(\Hp = S \mid \Mg = S) * P(\Zb = S \mid \Mg = S) \\
         & P(\Mg = N) * P(\Hp = S \mid \Mg = N) * P(\Zb = S \mid \Mg = N) \Big) /
           P(\Zb = S)\\
        =& ( 0.1*0.8*0.52 \ +\  0.9*0.7*0.088 ) / 0.1312 \\
        \approx & 0.739634146.
\end{align*}

\begin{question}
    Qual a probabilidade de se ver um zumbi após ver um hipogrifo?
\end{question}

Primeiro, computaremos a probabilidade de se ver um hipogrifo;
depois, é só utilizar o teorema de Bayes diretamente.

\begin{align*}
    P(\Hp = S) &= P(\Hp = S \mid \Mg = S) * P(\Mg = S) +
        P(\Hp = S \mid \Mg = N) * P(\Mg = N) \\
        &= 0.8*0.1 + 0.7*0.9 \\
        &= 0.71 \\
    P(\Zb = S \mid \Hp = S) &= \frac{P(\Hp = S \mid \Zb = S) * P(\Zb = S)}{P(\Hp = S)} \\
        &\approx 0.739634146 * 0.1312 / 0.71\\
        &\approx 0.136676056.
\end{align*}

\begin{question}
    Qual a probabilidade de se ver um zumbi
    após ver um viajante do tempo montado num hipogrifo?
\end{question}

Precisamos antes calcular a probabilidade de aparecer um viajante no tempo.
\begin{align*}
    P(\Tp = S) =& P(\Tp = S \mid \Vr = S) * P(\Vr = S) + {} \\
            & P(\Tp = S \mid \Vr = N) * P(\Vr = N) \\
            =& 0.3 * 0.2 + 0.1 *0.8 \\
            =& 0.14.
\end{align*}

Agora, procedendo como na questão 5
(e observando que $\Hp$ e $\Tp$ são independentes),
temos

\begin{align*}
    P(\Zb = S \mid \Tp = S \wedge \Hp = S) =& \Big(
        P(\Zb = S \wedge \Tp = S \wedge \Hp = S \wedge \Mg = S \wedge \Vr = S) + {} \\
      & P(\Zb = S \wedge \Tp = S \wedge \Hp = S \wedge \Mg = S \wedge \Vr = N) + {} \\
      & P(\Zb = S \wedge \Tp = S \wedge \Hp = S \wedge \Mg = N \wedge \Vr = S) + {} \\
      & P(\Zb = S \wedge \Tp = S \wedge \Hp = S \wedge \Mg = N \wedge \Vr = N)
        \Big) /\\
      & P( \Tp = S \wedge \Hp = S )
    \\
      = & \Big(
          P(\Mg = S) * P(\Hp = S \mid \Mg = S) * {}\\
        & P(\Zb = S \mid \Mg = S \wedge \Vr = S) * {}\\
        & P(\Vr = S) * P(\Tp = S \mid \Vr = S) \\
        & + \\
        & P(\Mg = S) * P(\Hp = S \mid \Mg = S) * {}\\
        & P(\Zb = S \mid \Mg = S \wedge \Vr = N) * {}\\
        & P(\Vr = N) * P(\Tp = S \mid \Vr = N) \\
        & + \\
        & P(\Mg = N) * P(\Hp = S \mid \Mg = N) * {}\\
        & P(\Zb = S \mid \Mg = N \wedge \Vr = S) * {}\\
        & P(\Vr = S) * P(\Tp = S \mid \Vr = S) \\
        & + \\
        & P(\Mg = N) * P(\Hp = S \mid \Mg = N) * {}\\
        & P(\Zb = S \mid \Mg = N \wedge \Vr = N) * {}\\
        & P(\Vr = N) * P(\Tp = S \mid \Vr = N)
        \Big) /\\
        & \big( P( \Tp = S ) * P( \Hp = S ) \big) \\
      = & (0.1*0.8*0.6 *0.2*0.3+{}\\
        & 0.1*0.8*0.5 *0.8*0.1+{}\\
        & 0.9*0.7*0.4 *0.2*0.3+{}\\
        & 0.9*0.7*0.01*0.8*0.1) / (0.71 * 0.14)\\
      \approx& 0.21835.
\end{align*}

\end{document}
