\documentclass{article}
\usepackage[utf8]{inputenc}
\usepackage[brazil]{babel}
\usepackage{amsmath}
\usepackage{amssymb}
\usepackage{tikz}
\usetikzlibrary{positioning}
\usepackage{booktabs}

\begin{document}

\title{
    INE5430 --- Inteligência Artificial \\
    Raciocínio Probabilístico --- Redes Bayesianas
}
\author{
    Tiago Royer - 12100776
}
\date{28 de Maio de 2015}
\maketitle

\section{Introdução}

O objetivo do trabalho é estudar o avistamento de zumbis no campus da UFSC.
Nos foi dado a rede bayesiana apresentada na figura \ref{rede_bayesiana}.

\begin{figure}[h]
    \makebox[\textwidth][c]{
        % A figura é mais larga que o texto,
        % portanto o comando padrão \centering não funciona.
        % Este makebox cria uma caixa que, do ponto de vista do resto do LaTeX,
        % terá largura \textwidth;
        % mas qualquer conteúdo dentro da caixa será alinhado ao centro,
        % por mais que extrapole a largura especificada.
        \begin{tikzpicture}
            \begin{scope}[every node/.style={circle, draw, minimum size = 1cm}]
                % Minimum size é para que todos os nós possuam o mesmo tamanho.
                \node (Hp) at (0, 0) {Hp};
                \node (Mg) at (1, 1.5) {Mg};
                \node (Zb) at (2, 0) {Zb};
                \node (Vr) at (3, 1.5) {Vr};
                \node (Tp) at (4, 0) {Tp};
            \end{scope}

            \begin{scope}[>=latex, thick]
                \draw[->] (Mg) -- (Hp);
                \draw[->] (Mg) -- (Zb);
                \draw[->] (Vr) -- (Zb);
                \draw[->] (Vr) -- (Tp);
            \end{scope}

            \begin{scope}[every node/.style={node distance = 1mm}]
                \node[above left = of Mg] {
                    \begin{tabular}{c c} \toprule
                        Mg = S & Mg = N \\ \midrule
                        0.1 & 0.9 \\ \bottomrule
                    \end{tabular}
                };

                \node[above right = of Vr] {
                    \begin{tabular}{c c} \toprule
                        Vr = S & Vr = N \\ \midrule
                        0.2 & 0.8 \\ \bottomrule
                    \end{tabular}
                };

                \node[left = of Hp] {
                    \begin{tabular}{c c c} \toprule
                        & Hp = S & Hp = N \\ \cmidrule{2-3}
                        Mg = S & 0.8 & 0.2 \\
                        Mg = N & 0.7 & 0.3 \\
                        \bottomrule
                    \end{tabular}
                };

                \node[right = of Tp] {
                    \begin{tabular}{c c c} \toprule
                        & Tp = S & Tp = N \\ \cmidrule{2-3}
                        Vr = S & 0.3 & 0.7 \\
                        Vr = N & 0.1 & 0.9 \\
                        \bottomrule
                    \end{tabular}
                };

                \node[below = 5mm of Zb] {
                    \begin{tabular}{c c c c} \toprule
                        & & Zb = S & Zb = N \\ \cmidrule{3-4}
                        Mg = S & Vr = S & 0.6 & 0.4 \\
                        Mg = S & Vr = N & 0.5 & 0.5 \\
                        Mg = N & Vr = S & 0.4 & 0.6 \\
                        Mg = N & Vr = N & 0.01 & 0.99 \\
                        \bottomrule
                    \end{tabular}
                };
            \end{scope}
        \end{tikzpicture}
    } % \makebox[\textwidth][c]
    \caption{
        Rede bayesiana usada como base no trabalho.
    }
    \label{rede_bayesiana}
\end{figure}

\end{document}
